
%\section{Data Overview} 

We describe here the SDSS dataset and seeing estimates used in this work. The
selected subset of data, the so-called Stripe 82, represents about one third of
all SDSS imaging data. 

\subsection{Stripe 82 dataset} 

The equatorial Stripe 82 region (22$^h$24$^m$ $<$ R.A. $<$ 04$^h$08$^m$, 
$-$1.27$^\circ$  $<$ Dec $<$ $+$1.27$^\circ$, about 
290 deg$^2$) from the southern Galactic cap ($-64^\circ < b <  -20^\circ$) was repeatedly imaged (of order
one hundred times) by SDSS to study time-domain phenomena (such as supernovae, asteroids, variable stars, quasar 
variability).  An observing stretch of SDSS imaging data is called a ``run''. Often there is only a single
run for a given observing night, though sometimes there are multiple
runs per night. In this paper we use seeing data for 
108 runs, with a total of 947,400 fields, obtained between September,
1998 and September 2008 (there are 6 camera columns, each with 5 filters; for more
details please see \citealt{Gunn2006}). All runs are obtained during the Fall observing season (September to 
December). Astrometric and photometric aspects of this dataset have been discussed in detail by 
\cite{Ivezic2007} and \cite{Sesar2007}. 


\subsection{The treatment of seeing in SDSS \label{sec:PSFdata}}
 
Even in the absence of atmospheric inhomogeneities, the SDSS telescope delivers images whose 
FWHMs vary by up to 15\% from one side of a CCD to the other; the worst effects are seen in 
the chips farthest from the optical axis \citep{Gunn2006}. Moreover, since the atmospheric 
seeing varies with time, the delivered image quality is a complex two-dimensional function 
even on the scale of a single field (for an example of the instantaneous image quality across 
the imaging camera, see Figure 7 in \citealt{SDSSEDR}). 
 
The SDSS imaging PSF is modeled 
heuristically in each band using a Karhunen-Lo\'{e}ve (K-L) transform \citep{Lupton2002}. 
Using stars brighter than roughly 20$^{th}$ magnitude, the PSF images from a series of five 
fields are expanded into eigenimages and the first three terms are kept (K-L transform is 
also known as the Principal Component Analysis). The angular variation of the eigencoefficients
is fit with polynomials, using data from the field in question, plus the immediately preceding 
and following half-fields. The success of this K-L expansion is gauged by comparing PSF 
photometry based on the modeled K-L PSFs with large-aperture photometry for the same 
(bright) stars \citep{SDSSEDR}. 
Parameters that characterize seeing for one field of imaging data are stored in the so-called psField 
files\footnote{https://data.sdss.org/datamodel/files/PHOTO\_REDUX/RERUN/RUN/objcs/CAMCOL/psField.html}. 
The status parameter flag for each field indicates the success of the K-L decomposition.

In addition to the K-L decomposition, the SDSS processing pipeline computes parameters of the 
best-fit circular double Gaussian to a PSF radial profile evaluated at the center of each field. 
The PSF radial profile is extracted by measuring the K-L image flux in a set of annuli, spaced 
approximately exponentially. Each annulus is divided into twelve 30$^\circ$ cells, and the 
variation of extracted counts is used to estimate the profile uncertainty in each annulus.  
The measured PSF profiles are extended to $\sim$30 arcsec using observations of bright stars 
and at such large radii double Gaussian fits underpredict the measured profiles. For this reason, 
the fits are extended to include the so-called ``power-law wings'', which is reminiscent of
the Moffat function,
\begin{equation}
\label{eq:SDSSPSF}
        PSF(r) = {\exp(-{r^2\over 2\,\sigma_1^2}) + b\,\exp(-{r^2\over 2\,\sigma_2^2})
           + p_0\left(1 + { r^2 \over \beta \sigma_P^2}\right)^{-\beta/2} \over 1 + b + p_0}.
\end{equation} 
The measured PSFs are thus modeled using 6 free parameters ($\sigma_1$, $\sigma_2$, $\sigma_P$,
$b$, $p_0$ and $\beta$), and the best-fit parameters are reported in the psField files. 
Given that the measured profiles include only up to 10 data points, the fits are usually excellent
although they do not appear very robust (for examples of bad fits see Sec.~\ref{sec:psffit}).
%Fig.~\ref{fig:psffit}). 

