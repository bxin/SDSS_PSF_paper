
% \section{Introduction}

The atmospheric seeing, the point-spread function (PSF) due to atmospheric turbulence, plays
a major role in ground-based astronomy \citep{Roddier1981}. An adequate description 
of the PSF is critical for photometry, star-galaxy separation, and for unbiased measures of 
the shapes of nonstellar objects \citep{Lupton2001}. In addition, better understanding of the 
PSF temporal variation can lead to improved seeing forecasts; for example, such forecasts are 
considered in the optimization of LSST observing strategy \citep{LSSToverview}.

Seeing varies with the wavelength of observation, and it also varies with time, on time 
scales ranging from minutes to years. These variations, as well as the radial seeing (PSF) 
profile, can be understood as manifestations of atmospheric instabilities due to turbulent layers. 
Although turbulence is a complex physical phenomenon, the basic properties of the atmospheric
seeing can be predicted from first principles \citep{Racine2009}. The
\vk~turbulence theory, 
an extension of the Komogorov theory that introduces a finite maximum 
size for turbulent eddies (the so-called
outer scale parameter), quantitatively predicts the seeing profile and the variation of seeing
with wavelength \citep{vk1, vk2}. Therefore, seeing measurements can be used to test the theory and estimate
the relevant physical parameters. 

An unprecedentedly large high-quality database of seeing measurements was delivered by the Sloan Digital Sky Survey (SDSS, \citealt{York2000}), a large-area multi-bandpass digital sky survey. The SDSS delivered homogeneous and deep 
($r\la22.5$) photometry in five bandpasses ($u$, $g$, $r$, $i$, and $z$, with effective wavelengths 
of 3551, 4686, 6166, 7480, and 8932 $\AA$), accurate to about 0.02 mag for unresolved sources 
not limited by photon statistics \citep{Sesar2007}. Astrometric positions are accurate to better 
than 0.1 arcsec per coordinate for sources with $r<20.5$ \citep{Pier2003}, and the morphological 
information from the images allows reliable star-galaxy separation to $r<21.5$ \citep{Lupton2002}.
 
The SDSS camera (Gunn et al. 1998) used drift-scanning observing mode (scanning along great circles
at the sidereal rate) and detected objects in the order 
$r$-$i$-$u$-$z$-$g$, with detections in two successive bands separated in time by 72 s. Each of the six camera
columns produces a 13.5 arcmin wide scan; the scans are split into ``fields'' 9.0 arcmin long, corresponding
to 36 seconds of time (the exposure time is 54.1 seconds because the sensor size is 2k by 2k pixels,
with 0.396 arcsec per pixel). 
The point-spread function (PSF) is estimated as a function of position
within each field, although we only use one estimate per field, and in each bandpass - there are about 
148 seeing estimates for each square degree of scanned sky. As a result, the SDSS measurements can
be used to explore the seeing dependence on time (on time scales from 1 minute to 10 hours) and 
wavelength (from the UV to the near-IR), as well as its angular correlation on the sky on scales from 
arcminutes to about 2.5 degrees. Thanks to large dynamic range for stellar brightness, the PSF can 
be traced to large radii ($\sim$30 arcsec) and compared to seeing profiles predicted by turbulence
theories\footnote{For most parts of this paper, we consider the SDSS PSF size
same as the seeing, since the PSF is dominated by the atmosphere. However,
the instrument also contributes to the PSF, as discussed in the next Section.}.

The SDSS seeing measurements represent an excellent database that has
not yet been systematically 
explored. Here we utilize about a million SDSS seeing estimates to study the seeing profile 
and its behavior as a function of time and wavelength, and compare our results to theoretical 
expectations. The outline of this paper is as follows. In \S2, we give a brief description of the 
observations and the data used in analysis. We describe the PSF profile analysis, 
including estimation method for the full-width-at-half-maximum (FWHM) seeing
parameter, in \S3. In \S4, we analyse the dependence of FWHM on wavelength, and its angular 
and temporal structure functions. We present and discuss our conclusions in \S5. 