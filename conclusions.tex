
%  \section{Discussion and Conclusions} 
 
The atmospheric seeing due to atmospheric turbulence plays a major role in 
ground-based astronomy; seeing varies with the wavelength of observation
and with time, on time scales ranging from minutes to years. Better empirical
and theoretical understanding of the seeing behavior can inform optimization 
of large survey programs, such as LSST. We have utilized here for the first 
time an unprecedentedly large database of about a million SDSS seeing estimates
and studied the seeing profile and its behavior as a function of time and wavelength.

We find that the observed PSF radial profile can be parametrized by only two parameters, 
the FWHM of a theoretically motivated profile and a normalization of the contribution 
of an empirically determined instrumental PSF. The profile shape is very similar to 
the ``double gaussian plus power-law wing'' decomposition used by SDSS image
processing pipeline, but here it is modeled with two free model parameters, rather 
than six as in SDSS pipeline (of course, the SDSS image processing pipeline had
to be designed with adequate flexibility to be able to efficiently and robustly 
handle various unanticipated behavior). We find that the PSF radial profile is well 
described by theoretical predictions based on both 
Kolmogorov and \vk's turbulence theory (see Fig.~\ref{fig:psffit}). Given the extra 
degree of freedom due to the instrumental PSF, the shape of the
measured radial profile alone 
is insufficient to reliably rule out either of the two theoretical profiles.  

We report empirical evidence that the wavelength dependence of the atmospheric 
seeing and its correlation with the seeing itself agrees better with the \vk~model 
than the Kolmogorov turbulence theory (see Fig.~\ref{fig:alpha_fwhm}). For example, 
the best-fit power-law index to describe the seeing wavelength dependence in conditions 
representative for LSST survey is much closer to $-0.3$ than to the usually assumed value
of $-0.2$ predicted by the Kolmogorov theory. 
We note that most of the long-term seeing statistics are measured at visible 
wavelengths. The knowledge of the wavelength-dependence of the seeing is useful 
for extrapolating the seeing statistics to other wavelengths, for example, to
the near-infrared where a lot of the adaptive optics programs operate. 
PSF-sensitive galaxy measurements require that the PSFs measured from
stars be interpolated both spatially and in color into galactic PSFs.

We have also measured the characteristic angular and temporal scales on which 
the seeing decorrelates. The angular structure function saturates at scales beyond 
0.5$-$1.0 degree. The seeing rms variation at large angular scales is about 5\%,
but we emphasize that our data do not probe scales beyond 2.5 degree. Comparisons 
with simulations of the LSST and PSF measurements at the CFHT site show good general 
agreement.

The power spectrum of the temporal behavior is found to be broadly consistent with 
a damped random walk model with characteristic timescale in the range $\sim5-30$ 
minutes, though data show a shallower high-frequency behavior. The high-frequency 
behavior can be quantitatively described by a single power law with index in the 
range $-1.5$ to $-1.0$. A hybrid model is likely needed to fully capture both the 
low-frequency and high-frequency behavior of the temporal variations of atmospheric
seeing. 

We conclude by noting that, while our numerical results may only apply to the SDSS 
site, they can be used as useful reference points when considering spatial and 
temporal variations of seeing at other observatories.
