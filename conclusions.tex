
%  \section{Discussion and Conclusions} 
 
Using SDSS Stripe 82 data, we have seen a direct evidence that the
wavelength dependence of the atmospheric seeing agrees better with the
\vk~model than the Kolmogorov turbulence theory.
Most of the long-term seeing statistics are taken at the visible
wavelength. The knowledge on the wavelength-dependence of the seeing
is useful for extrapolating the seeing statistics to other
wavelength, for example, the near-infrared, where a lot of the
adaptive optics programs operate.


We have also measured 
the characteristic angular and temporal scales on which the seeing
decorrelates.
The numerical results may only apply to the SDSS site, but can be used
as references when considering spatial and temporal variations of
seeing at other observatories.
Comparisons with simulations of the LSST and PSF measurements at CFHT
show good general agreement, taken into account that the results are
for different sites and different telescope optics.
The results can be useful for observatories in scheduling
observations to make best use of good seeing conditions.

We thank John Peterson for running the PhoSim simulations and
providing us with both the PhoSim and CFHT curves in
Fig.~\ref{fig:spatial}.

%check lognormal?
% - seeing distribution is  log–normal, due to the fact that seeing arises from the addition of uncorrelated random processes 

%check autocorrelation?
% - fig. 2: average autocorrelation function of seeing over 9 months:  A*exp(-t/tau) 
% see Eq. (3) in VMT 1998 for definition.
% This could be calculated using correlate_temporal_bin.py (autocor
% option), or make a copy of correlate_temporal_sf.py, and calculate
% in a similar way. But, in fig 2 of VMT 1998, why autocor(dt=0) not
% at 1?

%The seeing auto-correlation function
%\begin{equation}
%      C(\Delta t) = < \theta(t) \theta(t+\Delta t)>,
%\end{equation} 
%can be modeled as
%\begin{equation}
%      C(\Delta t) = A \, \exp(-\Delta t/\tau),
%\end{equation} 
%with $\tau \sim 1$ hour. 
 



% Statistics of turbulence profile at Cerro Tololo: 
%A. Tokovinin, S. Baumont and J. Vasquez
%Mon. Not. R. Astron. Soc. 340, 52–58 (2003)

%the Gemini site testing campaign at Cerro Pachon (Vernin et al. 2000; Avila et al. 2000)
%http://www.gemini.edu/documentation/webdocs/rpt/rpt-ao-g0094-1.ps
% SPIE: 
%http://proceedings.spiedigitallibrary.org/proceeding.aspx?articleid=898738

% Coulman (1985, ARAA 23, 19) for theory of turbulence and seeing in astronomy


% cite: Vernin, J.; Munoz-Tunon, C. 1992, Astronomy and Astrophysics, vol. 257, no. 2, p. 811-816. 

% Check for seeing forecastas and "flexible scheduling" !! 
% http://www.sciencedirect.com/science/article/pii/S1387647398000438
% also
% The temporal behaviour of seeing:
% Using a large amount of data gathered during previous seeing campaign
% at ORM, we analyse the temporal evolution of seeing in order to find
% out whether predictions could be made over a short time interval of a
% few hours. The first results are presented.
% http://www.sciencedirect.com/science/article/pii/S1387647398000517

% time dependence of seeing: http://adsabs.harvard.edu/abs/2001BASI...29...39S
% they give references for time scales (from 15 mins to 2 hours), also
% see a claim for 2 hour time scale:
% http://adsabs.harvard.edu/abs/2003A%26A...409.1169T
% TMT testing - seeing:
% http://adsabs.harvard.edu/abs/2009PASP..121.1151S 
% HSC seeing in V and K: bad seeing has flatter seeing vs. lambda due
% to outer scale effects
% http://adsabs.harvard.edu/abs/2016ExA....42...85O

% a good summary of work to date for introduction
% https://arxiv.org/pdf/1206.3319.pdf

% forecasting "These characteristics make forecasting seeing a tall challenge."
% http://adsabs.harvard.edu/abs/2015JPhCS.595a2029R
% but this one claims some success using ARIMA:
% http://adsabs.harvard.edu/abs/2016ExA....41..223K
