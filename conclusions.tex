
%  \section{Discussion and Conclusions} 
 
For now, a collection place of stuff from the seeing papers (commented out)



% Statistics of turbulence profile at Cerro Tololo: 
%A. Tokovinin, S. Baumont and J. Vasquez
%Mon. Not. R. Astron. Soc. 340, 52–58 (2003)

%the Gemini site testing campaign at Cerro Pachon (Vernin et al. 2000; Avila et al. 2000)
%http://www.gemini.edu/documentation/webdocs/rpt/rpt-ao-g0094-1.ps
% SPIE: 
%http://proceedings.spiedigitallibrary.org/proceeding.aspx?articleid=898738

% Coulman (1985, ARAA 23, 19) for theory of turbulence and seeing in astronomy

\begin{equation} 
FWHM = 0.98 {\lambda \over r_0} 
\end{equation} 
where $\lambda$ is the wavelength and $r_0$ is the Fried's
parameter. It can be shown that the variation of $r_0$ with wavelength
predicted by the Kolmogorov's turbulence implies $FWHM \propto \lambda^{-0.2}$. 	
% cite: Vernin, J.; Munoz-Tunon, C. 1992, Astronomy and Astrophysics, vol. 257, no. 2, p. 811-816. 

% Check for seeing forecastas and "flexible scheduling" !! 
% http://www.sciencedirect.com/science/article/pii/S1387647398000438
% also
% The temporal behaviour of seeing:
% Using a large amount of data gathered during previous seeing campaign
% at ORM, we analyse the temporal evolution of seeing in order to find
% out whether predictions could be made over a short time interval of a
% few hours. The first results are presented.
% http://www.sciencedirect.com/science/article/pii/S1387647398000517

% time dependence of seeing: http://adsabs.harvard.edu/abs/2001BASI...29...39S
% they give references for time scales (from 15 mins to 2 hours), also
% see a claim for 2 hour time scale:
% http://adsabs.harvard.edu/abs/2003A%26A...409.1169T
% TMT testing - seeing:
% http://adsabs.harvard.edu/abs/2009PASP..121.1151S 
% HSC seeing in V and K: bad seeing has flatter seeing vs. lambda due
% to outer scale effects
% http://adsabs.harvard.edu/abs/2016ExA....42...85O

% a good summary of work to date for introduction
% https://arxiv.org/pdf/1206.3319.pdf

% forecasting "These characteristics make forecasting seeing a tall challenge."
% http://adsabs.harvard.edu/abs/2015JPhCS.595a2029R
% but this one claims some success using ARIMA:
% http://adsabs.harvard.edu/abs/2016ExA....41..223K