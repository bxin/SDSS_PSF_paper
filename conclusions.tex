
%  \section{Discussion and Conclusions} 
 
For now, a collection place for stuff from other seeing papers (commented out),
and some text to move later to Sections 3 and 4. 


1) \cite{VMT1998} 

For seeing prediction, it is tempting to find statistical laws which describe its temporal behaviour.
To our knowledge only three authors (Racine, 1996; Munoz-Tunon et al., 1997; Sarazin, 1997) 
have worked on this problem. 

- fig. 2: average autocorrelation function of seeing over 9 months:  A*exp(-t/tau) 

- seeing distribution is  log–normal, due to the fact that seeing arises from the addition of uncorrelated random processes


with $\tau$=17 min and $\gamma \sim 0.7$. for observations at Mauna Kea \cite{Racine1996}

Seeing varies greatly with time and seems to decorrelate after 1 to 2 hr. 

The seeing auto-correlation function
\begin{equation}
      C(\Delta t) = < \theta(t) \theta(t+\Delta t)>,
\end{equation} 
can be modeled as
\begin{equation}
      C(\Delta t) = A \, \exp(-\Delta t/\tau),
\end{equation} 
with $\tau \sim 1$ hour. 
 
Triple correlation:
\begin{equation}
      C(\Delta t_1, \Delta t_2) = < \theta(t) \theta(t+\Delta t_1)\theta(t+\Delta t_2) >
\end{equation} 

2) 

The vertical distribution of the optical turbulence strength (energy profile), described by the altitude 
dependence of the refractive-index structure constant $C^2_n$, is hard to measure. When $C^2_n$ 
is available, the seeing $\theta$ is obtained by integrating $C^2_n$ over altitude $h$ as 
\begin{equation}
   \theta =  0.98 {\lambda \over r_0} = 5.25 \, \lambda^{-1/5} \, \left[ \int_0^\infty C_n^2(h) dh \right]^{3/5},
\end{equation}
where $r_0$ is the Fried's parameter \cite{Roddier1981}. 






% Statistics of turbulence profile at Cerro Tololo: 
%A. Tokovinin, S. Baumont and J. Vasquez
%Mon. Not. R. Astron. Soc. 340, 52–58 (2003)

%the Gemini site testing campaign at Cerro Pachon (Vernin et al. 2000; Avila et al. 2000)
%http://www.gemini.edu/documentation/webdocs/rpt/rpt-ao-g0094-1.ps
% SPIE: 
%http://proceedings.spiedigitallibrary.org/proceeding.aspx?articleid=898738

% Coulman (1985, ARAA 23, 19) for theory of turbulence and seeing in astronomy

\begin{equation} 
FWHM = 0.98 {\lambda \over r_0} 
\end{equation} 
where $\lambda$ is the wavelength and $r_0$ is the Fried's
parameter. It can be shown that the variation of $r_0$ with wavelength
predicted by the Kolmogorov's turbulence implies $FWHM \propto \lambda^{-0.2}$. 	
% cite: Vernin, J.; Munoz-Tunon, C. 1992, Astronomy and Astrophysics, vol. 257, no. 2, p. 811-816. 

% Check for seeing forecastas and "flexible scheduling" !! 
% http://www.sciencedirect.com/science/article/pii/S1387647398000438
% also
% The temporal behaviour of seeing:
% Using a large amount of data gathered during previous seeing campaign
% at ORM, we analyse the temporal evolution of seeing in order to find
% out whether predictions could be made over a short time interval of a
% few hours. The first results are presented.
% http://www.sciencedirect.com/science/article/pii/S1387647398000517

% time dependence of seeing: http://adsabs.harvard.edu/abs/2001BASI...29...39S
% they give references for time scales (from 15 mins to 2 hours), also
% see a claim for 2 hour time scale:
% http://adsabs.harvard.edu/abs/2003A%26A...409.1169T
% TMT testing - seeing:
% http://adsabs.harvard.edu/abs/2009PASP..121.1151S 
% HSC seeing in V and K: bad seeing has flatter seeing vs. lambda due
% to outer scale effects
% http://adsabs.harvard.edu/abs/2016ExA....42...85O

% a good summary of work to date for introduction
% https://arxiv.org/pdf/1206.3319.pdf

% forecasting "These characteristics make forecasting seeing a tall challenge."
% http://adsabs.harvard.edu/abs/2015JPhCS.595a2029R
% but this one claims some success using ARIMA:
% http://adsabs.harvard.edu/abs/2016ExA....41..223K