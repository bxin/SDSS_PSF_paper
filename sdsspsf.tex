\documentclass[12pt,preprint]{aastex}

\newcommand{\vdag}{(v)^\dagger}
\newcommand{\vk}{von K\'{a}rm\'{a}n}
\newcommand{\myemail}{skywalker@galaxy.far.far.away}
\slugcomment{Draft version}
 
\shorttitle{A Study of the Point Spread Function at SDSS}
\shortauthors{Xin, Ivezi\'{c} and Friends}
 
\begin{document}

\title{A Study of the Point Spread Function in SDSS Images}    %line break \\ is allowed in title

\author{B. Xin$^1$, \v{Z}. Ivezi\'{c}$^2$, R.H. Lupton$^3$, P. Yoachim$^2$, R.L. Jones$^2$, C. Claver$^1$, G. Angeli$^{1*}$}
\and
\author{Other Contributors}

\affil{$^1$Large Synoptic Survey Telescope, Tucson, AZ 85719}
\affil{$^2$Department of Astronomy, University of Washington, Seattle, WA 98195}
\affil{$^3$Department of Astrophysical Sciences, Princeton University, Princeton, NJ 08544}
\altaffiltext{*}{Current address: Giant Magellan Telescope
  Organization, Pasadena, CA 91107}

\begin{abstract}
We use SDSS imaging data in $ugriz$ passbands to study the shape of
point spread function (PSF) profile and the variation of its width with 
wavelength and time. We find that the PSF profile is well described by 
theoretical predictions based on \vk's turbulence theory. The observed 
profile can be parametrized by two parameters, the profile's FWHM (full width
at half maximum) and a normalization of the contribution of an empirical
instrumental PSF. The profile shape is very similar to the ``double gaussian
plus power-law wing'' decomposition used by SDSS, but here it is modeled 
with two free model parameters, rather than six as in SDSS pipeline. 
The FWHM variation with wavelength folows the
$\lambda^{\alpha}$ power law, where $\alpha \approx-0.3$, and is correlated
with the FWHM itself. This behavior is much better described by \vk's turbulence 
theory with the outer scale parameter in the range 10--30 m, than by the 
Kolmogorov's turbulence theory. We also measure the temporal and angular
structure functions for FWHM and compare them to simulations and
existing measurements.
The angular structure function saturates around 0.5$-$1.0 degree. 
The temporal behavior is found to be consistent with the damped random
walk model with a characteristic timescale of $\sim$ 15 minutes.
\end{abstract}

\keywords{imaging point spread function --- SDSS}
 


\section{Introduction}


% \section{Introduction}

The atmospheric seeing, the point-spread function (PSF) due to atmospheric turbulence, plays
a major role in ground-based astronomy. An adequate description of the PSF is critical for 
photometry, star-galaxy separation, and for unbiased measures of the shapes of nonstellar 
objects \citep{Lupton2001}. 

The vertical distribution of the optical turbulence strength, described by the altitude 
dependence of the refractive-index structure constant $C^2_n$, is hard to measure. 


The Sloan Digital Sky Survey (SDSS, \citealt{York2000}) is a large-area multi-bandpass sky survey. 
The SDSS provides homogeneous and deep ($r\la22.5$) photometry in five bandpasses ($u$, $g$, $r$, $i$, 
and $z$, with effective wavelengths of 3551, 4686, 6166, 7480, and 8932 $\AA$), 
accurate to about 0.02 mag for unresolved sources not limited by photon statistics \citep{Sesar2007}. 
Astrometric positions are accurate to better than 0.1 arcsec per coordinate for sources with $r<20.5$ 
\citep{Pier2003}, and the morphological information from the images allows reliable star-galaxy separation 
to $r<21.5$ \citep{Lupton2002}.
 
The SDSS camera (Gunn et al. 1998) used drift-scanning observing mode (scanning along great circles
at the sidereal rate) and detected objects in the order 
$r-i-u-z-g$, with detections in two successive bands separated in time by 72 s. Each of the six camera
columns produces a 13.5 arcmin wide scan; the scans are split into fields 9.0 arcmin long, corresponding
to 36 seconds of time (the exposure time is 54.1 seconds because the sensor size is 2k by 2k pixels,
with 0.396 arcsec per pixel). 
The point-spread function (PSF) is estimated for each field and in each bandpass - there are about 
148 seeing estimates for each square degree of scanned sky. As a result, the SDSS measurements can
be used to explore the seeing dependence on time (on time scales from 1 minute to 10 hours) and 
wavelength (from the UV to the near-IR), as well as its angular correlation on the sky on scales from 
arcminutes to about 2.5 degrees. Thanks to large dynamic range for stellar brightness, the PSF can 
be traced to large radii ($\sim$30 arcsec) and compared to seeing profiles predicted by turbulence
theories. 

The SDSS seeing measurements represent an excellent database that has not been systematically 
explored yet. Here we utilize about a million SDSS seeing estimates to study the seeing profile 
and its behavior as a function of time and wavelength, and compare our results to theoretical 
expectations. In addition to direct scientific drivers, we are also motivated by the possibility 
to use seeing forecasts to optimize the efficiency of LSST observing strategy \citep{LSSToverview}.


{\bf Bo or ZI:} add a brief summary with a few references (e.g. Young, Tyson, Roddier,
Tokovinin papers) about what is known regarding:  PSF profile shape (e.g. Kolmogorov vs. von Karman), 
the dependence of FWHM on wavelength, and angular and temporal structure functions. 
An expanded version can go to Discussion and Conclusions section. 


The outline of this paper is as follows. In \S2, we give a brief description of the 
observations and the data used in analysis. We describe the PSF profile analysis, 
including estimatation method for the full-width-at-half-maximum (FWHM) seeing
parameter, in \S3. In \S4, we analyse the dependence of FWHM on wavelength, and its angular 
and temporal structure functions. We discuss and present our conclusions in \S5. 

  

\section{Data Overview} 


%\section{Data Overview} 

We describe here the SDSS dataset and seeing estimates used in this work. The
selected subset of data, the so-called Stripe 82, represents about one third of
all SDSS imaging data. 

\subsection{Stripe 82 dataset} 

The equatorial Stripe 82 region (22$^h$24$^m$ $<$ R.A. $<$ 04$^h$08$^m$, 
$-$1.27$^\circ$  $<$ Dec $<$ $+$1.27$^\circ$, about 
290 deg$^2$) from the southern Galactic cap ($-64^\circ < b <  -20^\circ$) was repeatedly imaged (of order
one hundred times) by SDSS to study time-domain phenomena (such as supernovae, asteroids, variable stars, quasar 
variability).  An observing stretch of SDSS imaging data is called a ``run''. Often there is only a single
run for a given observing night, though sometimes there are multiple
runs per night. In this paper we use seeing data for 
108 runs, with a total of 947,400 fields, obtained between September,
1998 and September 2008 (there are 6 camera columns, each with 5 filters; for more
details please see \citealt{Gunn2006}). All runs are obtained during the Fall observing season (September to 
December). Astrometric and photometric aspects of this dataset have been discussed in detail by 
\cite{Ivezic2007} and \cite{Sesar2007}. 


\subsection{The treatment of seeing in SDSS \label{sec:PSFdata}}
 
Even in the absence of atmospheric inhomogeneities, the SDSS telescope delivers images whose 
FWHMs vary by up to 15\% from one side of a CCD to the other; the worst effects are seen in 
the chips farthest from the optical axis \citep{Gunn2006}. Moreover, since the atmospheric 
seeing varies with time, the delivered image quality is a complex two-dimensional function 
even on the scale of a single field (for an example of the instantaneous image quality across 
the imaging camera, see Figure 7 in \citealt{SDSSEDR}). 
 
The SDSS imaging PSF is modeled 
heuristically in each band using a Karhunen-Lo\'{e}ve (K-L) transform \citep{Lupton2002}. 
Using stars brighter than roughly 20$^{th}$ magnitude, the PSF images from a series of five 
fields are expanded into eigenimages and the first three terms are kept (K-L transform is 
also known as the Principal Component Analysis). The angular variation of the eigencoefficients
is fit with polynomials, using data from the field in question, plus the immediately preceding 
and following half-fields. The success of this K-L expansion is gauged by comparing PSF 
photometry based on the modeled K-L PSFs with large-aperture photometry for the same 
(bright) stars \citep{SDSSEDR}. 
Parameters that characterize seeing for one field of imaging data are stored in the so-called psField 
files\footnote{https://data.sdss.org/datamodel/files/PHOTO\_REDUX/RERUN/RUN/objcs/CAMCOL/psField.html}. 
The status parameter flag for each field indicates the success of the K-L decomposition.

In addition to the K-L decomposition, the SDSS processing pipeline computes parameters of the 
best-fit circular double Gaussian to a PSF radial profile evaluated at the center of each field. 
The PSF radial profile is extracted by measuring the K-L image flux in a set of annuli, spaced 
approximately exponentially. Each annulus is divided into twelve 30$^\circ$ cells, and the 
variation of extracted counts is used to estimate the profile uncertainty in each annulus.  
The measured PSF profiles are extended to $\sim$30 arcsec using observations of bright stars 
and at such large radii double Gaussian fits underpredict the measured profiles. For this reason, 
the fits are extended to include the so-called ``power-law wings'', which is reminiscent of
the Moffat function,
\begin{equation}
\label{eq:SDSSPSF}
        PSF(r) = {\exp(-{r^2\over 2\,\sigma_1^2}) + b\,\exp(-{r^2\over 2\,\sigma_2^2})
           + p_0\left(1 + { r^2 \over \beta \sigma_P^2}\right)^{-\beta/2} \over 1 + b + p_0}.
\end{equation} 
The measured PSFs are thus modeled using 6 free parameters ($\sigma_1$, $\sigma_2$, $\sigma_P$,
$b$, $p_0$ and $\beta$), and the best-fit parameters are reported in the psField files. 
Given that the measured profiles include only up to 10 data points, the fits are usually excellent
although they do not appear very robust (for examples of bad fits see Sec.~\ref{sec:psffit}).
%Fig.~\ref{fig:psffit}). 

 


\section{The PSF profile analysis}


%  \section{The PSF profile analysis}

Since the 6-parameter PSF fits was adopted by the SDSS processing
pipeline, significant progress has been made in validating the von
Karman model of the atmosphere and measuring the associated outer
scale. See, for example, \cite{Tokovinin2002},
\cite{Boccas2004}, and \cite{MartinezMessenger}.
In this section, we describe our 2-parameter fits to the SDSS PSF
profiles using the von Karman atmosphere model.

Our fits to each PSF profile is a 2-step process. First we fit the
measured PSF profile to a von Karman PSF, with only one free parameter -
the FWHM of the von Karman profile. Although in most cases, the fitted
curve agrees with the input data points very well, better than the
original 6-parameter double-Gaussian fit, it doesn't always describe
the PSF tail beyond $\sim 15$ arcsec radius. This is obvious, because
it is known that the PSF tails in the optical bands can be quite
different due to the properties of the CCDs.
Therefore, in the second step of our PSF modeling, we introduce an
empirical instrument PSF, so that the observed PSF can be expressed as
a convolution of the atmosphere, represented by the von Karman, and
the instrument PSF,

\begin{equation}
        PSF = vK (FWHM) \otimes PSF_{inst},
\end{equation} 
where
\begin{equation}
        PSF_{inst} = \exp(-\frac{r^2}{2\sigma^2}) + 10^{\eta(ar^2+br+c)}.
\label{eq:psfinst}
\end{equation} 
Because the shape of the instrument PSF tail should not vary with
time, the parameters $a$, $b$, and $c$ in Eq.~(\ref{eq:psfinst}) are
fixed for each band-camera-column combination.
Even though this second fit involves a 2-dimensional convolution,
there is only one free parameter in the fit - 
$\eta$, the normalization of the instrument PSF tail.
Each two-step PSF fit can be done in a few seconds.

%o. keep 4 data points only
%o. not using r0 as parameter
%o. normalization to 1.
%o. give tailPar?


Why is SDSS PSF different for the $i$ band? The $i$ band psf has ``stronger tails''
becuse of scattering in the CCD.  The Si is transparent at long $i$-band wavelengths 
so light goes all the way through the chip and is reflected off the solder, and passes 
back up through the Si. This effect is not visible in the $z$ band because in this case
thick front-side chips are used (in all other bands, thin back-side chips are used). 

Compare to SDSS, emphasize superiority of 1 parameter vs. 6 parameters fit

Discuss profile shape stability when the seeing is rapidly  changing 

\begin{figure}
\centering
\includegraphics[width=0.9\textwidth]{FIGURES/psffit.png}
\caption{Fit to the PSF profiles from run 4874, field 0. Red curves
  are results of 1-parameter von Karman fits. Blue curves are red
  curve convolved with the instrument PSF, where the scaling factor on
  the tail component is allowed to vary. Note that the y-axis is on
  logorithm scale.
\label{fig:psffit}}
\end{figure}
 

\section{The analysis of FWHM behavior} 


%  \section{The analysis of FWHM behavior} 

Given that the observed seeing is by and large described by a single parameter, FWHM, 
we study here three aspects of its variation in detail: dependence on wavelength,
the spatial (angular) structure function, and temporal behavior. We note that details 
about the seeing profile tails, including the contribution of the instrumental profile and
the $i$ band behavior, do not matter here because we focus only on the FWHM behavior. 

\subsection{The FWHM dependence on wavelength} 

The Kolmogorov turbulence theory gives a standard formula for the FWHM of a long-exposure
seeing-limited PSF in a large telescope~\citep{Roddier1981},
\begin{equation}
\textrm{FWHM}^{\rm Kolm}(\lambda, X) = \frac{0.976\lambda}{r_0(\lambda,X)},
\label{eq:fwhmkolm}
\end{equation}
\begin{equation}
r_0(\lambda,X) = r_0(\lambda_0, 1) \left(\frac{\lambda}{\lambda_0}\right)^{1.2}
\frac{1}{X^{0.6}},
\label{eq:r0}
\end{equation}
where $\lambda$ is the wavelength in meter, $X$ is the airmass,
$r_0$ is the Fried parameter in meter, and $\textrm{FWHM}^{\rm Kolm}$
is in radian.
We use $\lambda_0$ as the reference wavelength.
$r_0(\lambda_0, 1)$ is the $r_0$ for $\lambda=\lambda_0$ and $X$=1.
Substituting Eq.~(\ref{eq:r0}) into (\ref{eq:fwhmkolm}), it is easy to show that 
\begin{equation}
\textrm{FWHM}^{\rm Kolm} \propto \lambda^{-0.2}.
\end{equation}


With the \vk~atmosphere model, the FWHM as in
Eq.~(\ref{eq:fwhmkolm}) needs an additional correction factor
which is a function of the outer scale $L_0$~\citep{Tokovinin2002},
\begin{equation}
\label{eq:FWHMvK}
\textrm{FWHM}^{\rm vonK}(\lambda, X) = \frac{0.976\lambda}{r_0(\lambda,X)}
\sqrt{1-2.183\left( \frac{r_0(\lambda,X) }{L_0} \right)^{0.356}}.
\end{equation}
If a power-law approximation is attempted,  
\begin{equation}
\textrm{FWHM}^{\rm vonK} \propto \lambda^{\alpha},
\label{eq:fwhmvonk} 
\end{equation}
$\alpha$ becomes a function of $L_0$ and $r_0$ at a specified
wavelength and airmass, or equivalently, a function of $L_0$ and FWHM$^{\rm vonK}$.
For the subsequent analysis, we adopt the $r$ band as the fiducial band (with
the effective wavelength of 616.6 nm).


\begin{figure}[th]
\centering
\includegraphics[width=0.9\textwidth]{FIGURES/fwhm_lambda.png}
\caption{The behavior of FWHM as a function of wavelength for the fiducial run 4874.
Symbols are SDSS data and solid line is the best power-law fit, with the best-fit slope
($\alpha$) shown in inset. For comparison purposes, the $\alpha=-0.2$ (dotted) and $\alpha=-0.3$ 
(dashed) lines are also shown. For the ensemble behavior of best-fit $\alpha$, see Fig.~\ref{fig:alpha_fwhm}. 
\label{fig:fwhm_lambda}}
\end{figure}


\begin{figure}[th]
\centering
\includegraphics[width=0.5\textwidth]{FIGURES/alpha_fwhm.png}
\caption{The variation of the best-fit power-law index for the wavelength dependence of FWHM, $\alpha$, 
vs. the FWHM in the $r$-band for all the 108 Stripe 82 runs. The symbols are SDSS
measurements of $\alpha$ based on the $ugri$ data and averaged over camera columns 2 to 5. 
The curves are predictions of the 
\vk~model, with $L_0$ ranging from 2 meters to infinity, as labeled. The data are clearly
inconsistent with Kolmogorov predictions ($L_0=\infty$) and reasonably well described by
\vk~model and $L_0$ in the range from 5m to $\sim$100m.  \label{fig:alpha_fwhm}}
\end{figure}

 
For each run from SDSS Stripe 82 data, and each camera column, we make
a least-square fit to all the simultaneous FWHM measurements across the optical bands, to
estimate the power-law index $\alpha$ (see Eq.~\ref{eq:fwhmvonk}). 
The errors on the FWHM measurements in each optical band comes from
averaging over all the fields.
All FWHM values are multiplied by $1/X^{0.6}$ to 
correct for the airmass effects\footnote{The airmass dependence for the \vk~model 
is not strictly a power law, but can be approximated by a power law with good precision.
By numerically fitting a power law to eq.~\ref{eq:FWHMvK}, we obtained a power-law index 
of 0.63. We ignore the difference between 0.63 and 0.6 as it results in seeing variations 
below 1\% for the probed range of airmass.}.

All FWHM are normalized using 
corresponding FWHM in the $r$-band taken at the same moment in time. 
We take into account that the same field number does not correspond to the same
time in all filters. The scanning order in the SDSS camera is $r$-$i$-$u$-$z$-$g$, with the delay between the two 
successive filters corresponding to 2 fields. That is, if we take the field number $F$ for the $r$-band, then
we need to take FWHM for the $i$-band from field $F-2$, for the $u$-band
from $F-4$, and so on. 

Fig.~\ref{fig:fwhm_lambda} shows such fits for run 4874. Significant deviation 
from $\alpha = -0.2$, predicted by the Kolmogorov model, can be seen in most bands.
We find that fits in columns 1-5 are always similar, while in column 6 the slope is 
systematically lower. Similarly, the data in the $ugri$ bands are well fit by the power law, 
while the $z$ band the data are systematically larger than the power-law fit. 
For this reason, we refit the data using only $ugri$ bands and average results without
using the edge columns (1 and 6, though including column 1 does not substantially change 
the results). Fig.~\ref{fig:alpha_fwhm} shows a scatter plot of the resulting
best-fit $\alpha$ vs. the FWHM in the $r$-band, for all the analyzed runs.
 
As discussed above, according to the \vk~atmosphere model, the
power index $\alpha$ should be a function of the outer scale $L_0$ and 
FWHM. A correlation between $\alpha$ and the FWHM is discernible in
Fig.~\ref{fig:alpha_fwhm}. Similar correlations have been seen in Subaru images 
and reported by~\cite{subaruSeeing2016}.
The data points are overlaid with curves predicted by the 
\vk~model, with $L_0$ varying from 2 m to infinity.
The data clearly deviate from the Kolmogorov model prediction, which is
the horizontal line at $\alpha = -0.20$, with an infinite $L_0$.
For LSST's fiducial FWHM of 0.6 arcsec and the commonly assumed 
$L_0 = 30$ m, the \vk~model predicts an $\alpha$ value close to $-0.31$.



\subsection{Angular structure function} 

\begin{figure}[th]
\centering
\includegraphics[width=0.5\textwidth]{FIGURES/spatial.png}
\caption{The angular structure function for the PSF size determined using 
  CFHT data from \cite{heymans2012}, SDSS data analyzed here, and LSST image simulations. 
  SDSS measurements are averaged over 86 runs with the number of fields larger than 100. 
\label{fig:spatial}}
\end{figure}

To examine the angular (spatial) correlation of the FWHM, we compute the angular
structure function using PSF measurements from all 6 camera columns (always evaluated at 
the field center using the K-L expansion). 
Our structure function is defined as the root-mean-square scatter of the PSF size 
differences of pairs of stars in the same distance bin along the direction perpendicular
to the scanning direction\footnote{The adopted form 
of the structure function, $SF$, is closely related to the autocorrelation function, $ACF$, as 
$SF \propto (1-ACF)^{1/2}$.} .
The SDSS curves are combined for 86 Stripe 82 runs with the number of fields larger than 
100 (out of 108 runs) 
We also compared the structure functions for each band
separately, with and without camera columns 1 and 6, 
and found no statistically significant differences.
Results for the $r$-band are shown in Fig.~\ref{fig:spatial}.

The structure function starts saturating at separations of
$\sim 0.5 - 1.0$ degree, with an asymptotic value of about $\sim 0.05$ arcsec.
In other words, the seeing rms variation at large angular scales is about 5\%,
but we emphasize that our data do not probe scales beyond 2.5 degree. 

For comparison, Fig. ~\ref{fig:spatial} also shows results from the CFHT PSF 
measurements~\citep{heymans2012}, and simulated PSF angular
structure functions obtained using image simulation code PhoSim~\citep{phosim}. 
The PhoSim PSF profiles are obtained by simulating a grid of stars
spaced by 6 arcminutes with non-perturbed LSST telescope and ideal sensors.
The results are averaged over 9 different atmosphere realizations with
different wind and screen parameters and airmass, and over 3 different
wavelengths (350 nm, 660 nm, and 970 nm).
The CFHT PSF size measurements were made in the $i$-band, and provided
by the authors of~\cite{heymans2012}.
The three curves in Fig.~\ref{fig:spatial} appear to be quantitatively
consistent with each other, even though they correspond to telescopes at
different sites and with different optics. 

We note that the PhoSim code could be used to further quantitatively study the variation
of seeing with outer scale and the impact of telescope diameter; however,
such detailed modeling studies are beyond the scope of this report. 


\subsection{Temporal behavior}

\subsubsection{Power spectrum analysis} 

\begin{figure}[th]
\centering
\includegraphics[width=0.99\textwidth]{FIGURES/temporalPSD.png}
\vskip -0.2in
\caption{PSF size temporal power spectral density for run 4874, r-band. 
The solid lines are fits using the damped random walk model. 
The dashed lines show best fits based on a single power law. The former
predicts a steeper high-frequency behavior, while the latter cannnot 
explain the turnover at low frequencies. 
\label{fig:psd}}
\end{figure}

To study the temporal behavior of the seeing, we first analyze its power spectrum.
Fig.~\ref{fig:psd} shows the temporal power spectral density (PSD) of the
PSF FWHM for 6 camera columns, in run 4874, $r$-band.
The time difference between subsequent fields is 36 seconds. 
Even though anormalies on the wavelength dependence of the FWHM are
seen in column 6, it is clear from Fig. 6 that the temporal behavior
of the FWHM does not vary with band. The temporal analysis presented
in this section includes all the camera columns and all the optical
bands. We have repeated the analysis without using FWHM measurements
from $z$-band and camera columns 1 and 6, and found no statistically significant
differences in results.

We fit the PSD using two competing models.
The first is a damped random walk (DRW) model~\citep[for introduction see Chapter 10 in][]{zeljkoBook},
\begin{equation}
\textrm{PSD}(f) = \frac{\tau^2 SF^2_{\infty}}{1+(2\pi f \tau)^2},
\end{equation}
where $f$ is the temporal frequency, $SF_{\infty}$ is the asymptotic
value of the structure function, and $\tau$ is the
characteristic timescale.
The solid curves in Fig.~\ref{fig:psd} show fits using this model,
with $f$, $\tau$, and $SF_{\infty}$ as free parameters.
Note that due to the lack of data toward the low-frequency end, the
first and second bins are four and two times wider than the
rest of the bins, respectively.
Combining fit results for all camera columns and optical bands for run 4874
gives $\tau = 23.6 \pm 1.3$ minutes.
Making the same fits for all the 108 runs in Stripe 82, 
we obtain the $\tau$ distribution vs. the duration of each
run, as shown in Fig.~\ref{fig:hist} (left).
The shorter runs tend to give smaller timescale. It is plausible that short runs 
cannot reliably constrain $\tau$ due to the lack of data toward the low-frequency 
end of the spectra. There are 12 runs longer than 6 hours and their characteristic timescales
are within the range of about $\sim5-30$ minutes.
This result is generally consistent with ~\cite{Racine1996}, where a timescale of 
$\tau = 17 \pm 1$ minutes was found.

The data consistently show a shallower high-frequency behavior than predicted
by damped random walk ($\propto 1/f^2$). In order to quantitatively describe 
the high-frequency tail of the PSD, we fit a simple power law,
\begin{equation}
\textrm{PSD}(f) = B f^\beta,
\end{equation}
where $B$ is the normalization factor, and $\beta$ is the power-law index.
Best-fits are illustrated for run 4874 are in Fig.~\ref{fig:psd} (dashed lines).
Combining fit results for all camera columns and filters gives $\beta = -1.29\pm 0.09$ 
for run 4874. Making the same fits for all the 108 runs in Stripe 82, we obtained the 
$\beta$ distribution vs. the duration of each run shown in Fig.~\ref{fig:hist} (right).
The shorter runs give $\beta$ values with a larger variance, but nevertheless it is 
clear that for most runs the high-frequency behavior can be described with a 
power-law index in the range $-1.5$ to $-1.0$. 
We note that \cite{2016SPIE.9906E..42S} measured steeper slopes,
though at much higher frequencies.
On the other hand, a single 
power law cannot explain the turnover at low frequencies. 

Therefore, neither model provides a satifactory fit over the entire frequency range: 
the power law fit systematically over predicts the low-frequency part of the PSD,
while the $1/f^2$ high-frequency behavior of damped random walk model is too 
steep. It is likely that a hybrid model would work, for example, a simple
generalization of the random walk model
(see Eq. (19) in \cite{Dunkley2005}), 
\begin{equation}
\textrm{PSD}(f) = \frac{\tau^\alpha SF^2_{\infty}}{1+(2\pi f
  \tau)^\alpha}.
\label{eq:hybrid}
\end{equation}
Performing fits to our data using this model did not yield useful
results on the characteristic timescale, due to our lack of data
points at low frequency, and therefore the incapability in
constraining one additional parameter ($\alpha$ in Eq.~(\ref{eq:hybrid})).
We leave more detailed analysis,
perhaps informed by the PhoSim modeling, for future work. 


\begin{figure}[th]
\centering
\includegraphics[width=0.7\textwidth]{FIGURES/taubeta.png}
\vskip -0.2in
\caption{Left: The symbols show the best-fit characteristic timescale $\tau$ in 
damped random walk for all 108 runs in Stripe 82 vs. the duration of 
each run. It is plausible that short runs cannot reliably constrain
$\tau$.
The vertical dashed line indicates run duration of 6 hours.
Right: The power-law index $\beta$ for a single power law fit for all 108 
runs vs. the duration of each run. Note that for the majority of runs, $\beta$
is larger than the value appropriate for damped random walk ($\beta = -2$). 
\label{fig:hist}}
\end{figure}



\subsubsection{Structure function analysis} 


\begin{figure}[th]
\centering
\includegraphics[width=0.7\textwidth]{FIGURES/fdt.png}
\caption{Left: The average normalized seeing difference, $\langle f(\Delta t)\rangle$, as
  a function of the time separation, $\Delta t$, for run 4874, camera
   column 1, in the $r$-band. The fit to Eq.~(\ref{eq:fdt}) gives $f(\Delta t) ^\infty =  
   0.088\pm0.005$, $\tau^\ast = 45.1\pm10.3$ minutes and $\gamma$ =
   1.016$\pm$0.102. 
A damped random walk model has $\gamma=1$.
%The dashed line is the 
%   structure function predicted by damped random walk model with the same 
%   $f(\Delta t) ^\infty$ and $\tau$. 
Right: The timescale $\tau^\ast$ vs. the duration of each run.
 Note that for most runs $\tau^\ast$ is in the range 5-30 minutes
    (runs shorter than 20 minutes or where the fitted $\tau^\ast$ is longer than 2/3 of the duration
    of the run are left out).  
\label{fig:fdt}}
\end{figure}


An alternative approach to power spectrum analysis is offered by auto-correlation 
and structure function analysis. Following~\cite{Racine1996}, we define a 
structure-function-like quantity
\begin{equation}
       f(\Delta t) = {| \theta(t+\Delta t) - \theta(t)| \over  \theta(t+\Delta t) + \theta(t) },
\end{equation} 
where $\theta$ is seeing. We then fit the mean value of $f(\Delta t)$ with the following function
\begin{equation}
    \langle f(\Delta t) \rangle =  f(\Delta t) ^\infty \, \left( 1 - \exp[-(\Delta
      t/\tau^\ast)^\gamma] \right)^{1/2},
\label{eq:fdt}
\end{equation} 
with $f(\Delta t) ^\infty$, $\tau^\ast$ and $\gamma$ as free parameters.
Fig.~\ref{fig:fdt} (left) shows one example of such fits. This functional form is 
somewhat inspired by damped random walk model, where $\gamma=1$ and the 
term in brackets is raised to the power of a half. 
For this particular fit in Fig.~\ref{fig:fdt} (left), $\gamma$ is very close to 1.
%For illustration, Fig.~\ref{fig:fdt} 
%also shows a damped random walk model fit with the same $f(\Delta t) ^\infty$
%and $\tau$. 

The best-fit $\gamma$ is found to be mostly in the range 0.5 -1.5. 
The distribution of $\tau^\ast$ vs. 
the duration of each run is shown in Fig.~\ref{fig:fdt} (right) (somewhat 
arbitrarily,
runs shorter than 20 minutes and those where the fitted $\tau^\ast$ is longer than 2/3 of the duration
    of the run are deemed unreliable and not shown).  
%While the short runs show a larger variance in $\tau$, 
It is evident that 
for most runs the timescale $\tau^\ast$ is in the range 5-30 minutes.
Therefore, this analysis seems more robust at constraining the characteristic time
scale than fitting a damped random walk model to empirical PSD. 
 

\section{Discussion and Conclusions} 


%  \section{Discussion and Conclusions} 
 
For now, a collection place for stuff from other seeing papers (commented out),
and some text to move later to Sections 3 and 4. 


1) \cite{VMT1998} 

For seeing prediction, it is tempting to find statistical laws which describe its temporal behaviour.
To our knowledge only three authors (Racine, 1996; Munoz-Tunon et al., 1997; Sarazin, 1997) 
have worked on this problem. 

- fig. 2: average autocorrelation function of seeing over 9 months:  A*exp(-t/tau) 

- seeing distribution is  log–normal, due to the fact that seeing arises from the addition of uncorrelated random processes

Following Racine (1996), they define a structure-function-like quantity
\begin{equation}
       f(\Delta t) = {| \theta(t+\Delta t) - \theta(t)| \over  \theta(t+\Delta t) + \theta(t) },
\end{equation} 
where $\theta$ is seeing. The mean value is
\begin{equation}
    < f(\Delta t) > =  f(\Delta t) ^\infty \, \left[ 1 - \exp(-\Delta t/\tau)^\gamma \right],
\end{equation} 
with $\tau$=17 min and $\gamma \sim 0.7$. for observations at Mauna Kea \cite{Racine1996}

Seeing varies greatly with time and seems to decorrelate after 1 to 2 hr. 

The seeing auto-correlation function
\begin{equation}
      C(\Delta t) = < \theta(t) \theta(t+\Delta t)>,
\end{equation} 
can be modeled as
\begin{equation}
      C(\Delta t) = A \, \exp(-\Delta t/\tau),
\end{equation} 
with $\tau \sim 1$ hour. 
 
Triple correlation:
\begin{equation}
      C(\Delta t_1, \Delta t_2) = < \theta(t) \theta(t+\Delta t_1)\theta(t+\Delta t_2) >
\end{equation} 

2) 

The vertical distribution of the optical turbulence strength (energy profile), described by the altitude 
dependence of the refractive-index structure constant $C^2_n$, is hard to measure. When $C^2_n$ 
is available, the seeing $\theta$ is obtained by integrating $C^2_n$ over altitude $h$ as 
\begin{equation}
   \theta =  0.98 {\lambda \over r_0} = 5.25 \, \lambda^{-1/5} \, \left[ \int_0^\infty C_n^2(h) dh \right]^{3/5},
\end{equation}
where $r_0$ is the Fried's parameter \cite{Roddier1981}. 






% Statistics of turbulence profile at Cerro Tololo: 
%A. Tokovinin, S. Baumont and J. Vasquez
%Mon. Not. R. Astron. Soc. 340, 52–58 (2003)

%the Gemini site testing campaign at Cerro Pachon (Vernin et al. 2000; Avila et al. 2000)
%http://www.gemini.edu/documentation/webdocs/rpt/rpt-ao-g0094-1.ps
% SPIE: 
%http://proceedings.spiedigitallibrary.org/proceeding.aspx?articleid=898738

% Coulman (1985, ARAA 23, 19) for theory of turbulence and seeing in astronomy

\begin{equation} 
FWHM = 0.98 {\lambda \over r_0} 
\end{equation} 
where $\lambda$ is the wavelength and $r_0$ is the Fried's
parameter. It can be shown that the variation of $r_0$ with wavelength
predicted by the Kolmogorov's turbulence implies $FWHM \propto \lambda^{-0.2}$. 	
% cite: Vernin, J.; Munoz-Tunon, C. 1992, Astronomy and Astrophysics, vol. 257, no. 2, p. 811-816. 

% Check for seeing forecastas and "flexible scheduling" !! 
% http://www.sciencedirect.com/science/article/pii/S1387647398000438
% also
% The temporal behaviour of seeing:
% Using a large amount of data gathered during previous seeing campaign
% at ORM, we analyse the temporal evolution of seeing in order to find
% out whether predictions could be made over a short time interval of a
% few hours. The first results are presented.
% http://www.sciencedirect.com/science/article/pii/S1387647398000517

% time dependence of seeing: http://adsabs.harvard.edu/abs/2001BASI...29...39S
% they give references for time scales (from 15 mins to 2 hours), also
% see a claim for 2 hour time scale:
% http://adsabs.harvard.edu/abs/2003A%26A...409.1169T
% TMT testing - seeing:
% http://adsabs.harvard.edu/abs/2009PASP..121.1151S 
% HSC seeing in V and K: bad seeing has flatter seeing vs. lambda due
% to outer scale effects
% http://adsabs.harvard.edu/abs/2016ExA....42...85O

% a good summary of work to date for introduction
% https://arxiv.org/pdf/1206.3319.pdf

% forecasting "These characteristics make forecasting seeing a tall challenge."
% http://adsabs.harvard.edu/abs/2015JPhCS.595a2029R
% but this one claims some success using ARIMA:
% http://adsabs.harvard.edu/abs/2016ExA....41..223K 


\acknowledgments




%% To help institutions obtain information on the effectiveness of their
%% telescopes, the AAS Journals has created a group of keywords for telescope
%% facilities. A common set of keywords will make these types of searches
%% significantly easier and more accurate. In addition, they will also be
%% useful in linking papers together which utilize the same telescopes
%% within the framework of the National Virtual Observatory.
%% See the AASTeX Web site at http://aastex.aas.org/
%% for information on obtaining the facility keywords.

%% After the acknowledgments section, use the following syntax and the
%% \facility{} macro to list the keywords of facilities used in the research
%% for the paper.  Each keyword will be checked against the master list during
%% copy editing.  Individual instruments or configurations can be provided 
%% in parentheses, after the keyword, but they will not be verified.

{\it Facilities:} \facility{SDSS}, \facility{LSST}.

%% Appendix material should be preceded with a single \appendix command.
%% There should be a \section command for each appendix. Mark appendix
%% subsections with the same markup you use in the main body of the paper.

%% Each Appendix (indicated with \section) will be lettered A, B, C, etc.
%% The equation counter will reset when it encounters the \appendix
%% command and will number appendix equations (A1), (A2), etc.

%\appendix
%
%\section{Appendix material}

%% The reference list follows the main body and any appendices.
%% Use LaTeX's thebibliography environment to mark up your reference list.
%% Note \begin{thebibliography} is followed by an empty set of
%% curly braces.  If you forget this, LaTeX will generate the error
%% "Perhaps a missing \item?".
%%
%% thebibliography produces citations in the text using \bibitem-\cite
%% cross-referencing. Each reference is preceded by a
%% \bibitem command that defines in curly braces the KEY that corresponds
%% to the KEY in the \cite commands (see the first section above).
%% Make sure that you provide a unique KEY for every \bibitem or else the
%% paper will not LaTeX. The square brackets should contain
%% the citation text that LaTeX will insert in
%% place of the \cite commands.

%% We have used macros to produce journal name abbreviations.
%% AASTeX provides a number of these for the more frequently-cited journals.
%% See the Author Guide for a list of them.

%% Note that the style of the \bibitem labels (in []) is slightly
%% different from previous examples.  The natbib system solves a host
%% of citation expression problems, but it is necessary to clearly
%% delimit the year from the author name used in the citation.
%% See the natbib documentation for more details and options.


\bibliographystyle{aj}
\bibliography{sdsspsf}


%% The following command ends your manuscript. LaTeX will ignore any text
%% that appears after it.

\end{document}

%%
%% End of file `sample.tex'.

----------Completed: ----------------

ZI:  find exact definition of neff_psf (is it based on fit or input data?) 
Bo: add analysis directory and py code 


1)  make files, one per run, as a function of field number, with
- field, camera column, bandpass, 
- one-parameter fit FWHM from fitting \vk profile
- other SDSS params: psf_width, airmass, mjd, psf_nstar, neff_psf, sky_frames 


E.g. for some run
#  field  camCol  filter   FWHMvK    (other SDSS params)

2) plot FWHMvK vs. psf_width for 30 panels 

-----use a Gaussian as convolution kernel, remake fits -------

Bo:
check neff_psf from the vK fit and compare to 2G fit.
(checked fwhmeff, i.e., fwhmvk, and compared to psf_width from the 2G
fit. Generally the agreement is good. When they don't agree very well,
visually checking the vK fit vs the 2G fit shows that the vK fit often
gives the more reasonable agreement between data and fit.)
------make RMS (x - data) plots to quantify -----------
            /n(-1?) for 1-4, 5+ separately
            exclude bad 2G fits

3) plot alpha (index for lambda dep.) histograms for all data and for 
    two realizations of ``longest stretch'' selection
(not very conclusive, due to z-band tipping upward. The i-band could
be problematic too)

----------to do ----------------

---alpha vs seeing, col#6, col#3
??? masterTXT, a column to flag failed 2D fits?
???(in the master txt file for alpha, add seeing as a column)

4) plot profiles for a range of fields when the seeing is rapidly changing (is profile shape changing?) 


-------------Analysis structure: -----------------

1) profile discussion
- compare to SDSS, emphasize 1 parameter vs. 6 parameters
- discuss profile shape stability when seeing is changing rapidly

2) dependence on wavelength

- what is alpha distribution? 

3) cross-correlation of camera columns and measurement of structure function
    (that is, the covariance vs. angular distance) 

---for each run, each filter, make plot of cov vs separation
----fit through (0,1), plot slope as a function of seeing

4) auto-correlation function (and power spectrum) for temporal variation

- let's start with plain Fourier transform and look at power spectra (we can 
    play games with co-addition, 6 columns for a given run, or all runs)
- potentially, we could also look at  Kelly and Becker code  (2014, ApJ 788, 33) 
- here we can compare to Chuck's CP measurements (in opsim db) 

---for each run, each filter, camcol (#2,3), make D vs frequency.


Adaptive Beaming and Imaging in the Turbulent Atmosphere 
by Vladimir P. Lukin; Boris V. Fortes, Chapter 3, SPIE Press Book.
Fig 3: the depdendence of FWHM on D, Lo and lambda 




FWHM vs. field analysis

1040: 122 fields, oscillations on 5 min timescale, look at PSD 
2583: 227 fields, excellent seeing (1.1 in r) 
2662: 469 fields, long with typical seeing, with worst at the end
2709: 263 fields, oscillations on 5 min timescale, look at PSD 
2728: 488 fields, long with good seeing (1.2 in r) 
3325: 493 fields, long with typical seeing, with worst at the start
3360: 513 fields, typical but very variable seeing
3384: 784 fields, a long and *very* oscillatory run
3388: 713 fields, very different PSD than for 3384 
4145: 505 fields, steady deterioration from 1 arcsec to ~2 arcsec
4198: 747 fields, another long run with typical and oscillatory seeing
4203: 777 fields, another long run
4207: 753 fields, another long run
4849: 918 fields, a long run, typical seeing 
4868: 609 fields, another long run
4874: 981 fields, the longest run, typical seeing   *poster child* 


I would exclude runs with fewer than 100 fields from our analysis 